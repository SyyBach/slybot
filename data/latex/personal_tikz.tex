\usepackage{pgf,tikz}
\usepackage{ifthen}
\usetikzlibrary{arrows}
\usetikzlibrary{through}
\usetikzlibrary{intersections}
\usetikzlibrary{calc}
\usetikzlibrary{patterns}


\def\epsRad{0.85} % default value, always used as a radii info so far ! -> can be redef as length
\newlength{\tmplength}
\newlength{\figWidth}


\makeatletter
\newcommand*{\DivideLength}[3]{% length number1 number2 -> length*number2/number1
  \dimexpr\number\numexpr\number\dimexpr#1\relax*\number#3/\number#2\relax sp\relax
}
\makeatother



%
%% Macros

%% Very local macros
% Midle arrow segment
\newcommand*{\myArrow}[3]%
{ % Usage: \myArrow{start}{end}{arrowPos}
  \path (#1)--(#2) node (arrow#1#2pos) [pos=#3,inner sep=0pt,outer sep=0pt] {};
  \draw [->,shorten >=-1.5pt] (#1)--(arrow#1#2pos);
  \draw (arrow#1#2pos)--(#2);
}


%
% Circle draw
\newcommand*{\circleDraw}[2]%
{ % Usage: \circleDraw{center}{point}   --> use trueCirclePath instead
  \node [circle through={(#2)},color=myGray,draw,loosely dashed] at (#1) {} ;
}


% New circle path
\newcommand*{\circlePath}[2]%
{ % Usage: \circlePath{center}{point}
  \node [circle through={(#2)},name path=b#1#2] at (#1) {} ;
}
% Intersection of circle path should be reworked with default argument
\newcommand*{\circleIntersect}[3]%
{
  \path [name intersections={of=#1 and #2,sort by=#1}];
  \coordinate (#3) at (intersection-1);
}
\newcommand*{\circleIntersectBis}[3]%
{
  \path [name intersections={of=#1 and #2}];
  \coordinate (#3) at (intersection-2);
}
%

\makeatletter      
%
\newif\if@mydebug % used to control the drawing of control points for some commands below
%\@mydebugtrue
%
% Projects bd vtx onto inner projections
\newcommand*{\myIntersection}[2]%
{ % Usage: \myIntersection{vtx}{inner}
  \path [name path=seg#1#2] (#1)--(#2);
  \path [name path=circ#1] (#1) circle (\epsRad);
  \path [name intersections={of=seg#1#2 and circ#1,total=\t}]
    \pgfextra{\global\let\@mytotal\t};
  \ifnum\@mytotal>0
    \coordinate (#1#2) at (intersection-1);
    \if@mydebug
      \fill (#1#2) circle [radius=1.5pt,color=black] ;
      \node at (#1#2) [above] {#1#2};
    \fi
  \fi
}
%
% Computes intersection between two sop's
\newcommand*{\sopIntersection}[2]%
{ % Usage: \sopIntersection{vtx1}{vtx2}
  \path [name path=sop#1] (#1) circle (\epsRad);
  \path [name path=sop#2] (#2) circle (\epsRad);
  \path [name intersections={of=sop#1 and sop#2,total=\t}]
    \pgfextra{\global\let\@mytotal\t};
  \ifnum\@mytotal>0
    % Recompute with appropriate sorting. Cannot sort when empty (crash)
    \path [name intersections={of=sop#1 and sop#2,sort by=sop#1}];
    \foreach \s in {1,...,\@mytotal}
    {
      \coordinate (#1#2-\s) at (intersection-\s);
      \if@mydebug
        \fill (#1#2-\s) circle [radius=1.5pt,color=black] ;
        \node at (#1#2-\s) [above] {#1#2-\s};
      \fi
    }
  \fi
}
\makeatother
%%/Very local macros






%% More general purpose macros

%
% Compute angle formed by a vector with the X-axis
\newcommand{\getAngle}[3]%
  % Usage:
  % \getAngle{vecOrigin}{vecDest}{angleRegister}
{
  \pgfmathanglebetweenpoints{\pgfpointanchor{#1}{center}}
                            {\pgfpointanchor{#2}{center}}
  \global\let#3\pgfmathresult % we need a global macro 
}



%
% Computes length of a vector
\makeatletter
\newcommand{\getLength}[3]
{ % Usage:
  % \getLength{vecOrigin}{vecDest}{lengthRegister}
  \pgfpointdiff{\pgfpointanchor{#1}{center}}
               {\pgfpointanchor{#2}{center}}
  \pgf@xa=\pgf@x % no need to use a new dimen
  \pgf@ya=\pgf@y
  \pgfmathparse{veclen(\pgf@xa,\pgf@ya)/\strip@pt\tmplength} % to convert from pt to cm   
  %\pgfmathparse{veclen(\pgf@xa,\pgf@ya)/28.45274} % to convert from pt to cm   
  \global\let#3\pgfmathresult % we need a global macro
  % note to self: the result register will only contain a number, which can be used (most
  % likely directly?) as a relative length\ldots As of now the scale is set to cm so not
  % entirely sure of that
}


%
% Internal command to encapsulate the necessary computations for arc-drawing macros
\newcommand{\@arcCompute}[3]
{ % Usage:
  % \@arcCompute{center}{rightPoint}{leftPoint}
  % It is assumed that both point are at the same distance from the center
  \getLength{#1}{#2}{\@myArcRadius}%
  \getAngle{#1}{#2}{\@myRightAngle}%
  \getAngle{#1}{#3}{\@myLeftAngle}%
  \ifthenelse{\lengthtest{\@myRightAngle pt < \@myLeftAngle pt}}
  { % true, do nothing
  }
  { % false, increase leftAngle by 360
    \pgfmathparse{\@myLeftAngle+360}
    \global\let\@myLeftAngle\pgfmathresult
  }
}


%
% Macros for drawing arcs
% Usage: \macro{center}{start}{end}
\def\arcMacro#1#2#3{%
  \pgfextra{\@arcCompute{#1}{#2}{#3}}%
  arc(\@myRightAngle:\@myLeftAngle:\@myArcRadius)
  -- (#3) % required to arrive precisely at point #3, otherwise numerical imprecision stacks up
}
%
\def\reverseArcMacro#1#2#3{%
  \pgfextra{\@arcCompute{#1}{#3}{#2}}%
  arc(\@myLeftAngle:\@myRightAngle:\@myArcRadius)
  -- (#3) % required to arrive precisely at point #3, otherwise numerical imprecision stacks up
}
%
\makeatother


%
% Macro for a ``true'' circle path, to use in path/clipping/draw/fill (center->bd point)
% mostly untested in complex environment, probably needs to be the first path command
% redundant with the circle through lib, but at least this is not a node but a path command
\def\trueCirclePath#1#2{%
  let \p1 = ($(#1)-(#2)$), \n1 = {veclen(\p1)} in
  (#1) circle (\n1)
}
%%/More general purpose macros


%
% Utility to get coordinate to work with in pgf
\newdimen\XCoord
\newdimen\YCoord
\newcommand*{\ExtractCoordinate}[1]{\path (#1); \pgfgetlastxy{\XCoord}{\YCoord};}

%%/Macros


%% Custom diagonal line patterns
\makeatletter
\pgfdeclarepatternformonly[\LineSpace]{my north east lines}{\pgfqpoint{-1pt}{-1pt}}{\pgfqpoint{\LineSpace}{\LineSpace}}{\pgfqpoint{\LineSpace}{\LineSpace}}%
{
  \pgfsetcolor{\tikz@pattern@color}
  \pgfsetlinewidth{0.4pt}
  \pgfpathmoveto{\pgfqpoint{0pt}{0pt}}
  \pgfpathlineto{\pgfqpoint{\LineSpace + 0.1pt}{\LineSpace + 0.1pt}}
  \pgfusepath{stroke}
}

\pgfdeclarepatternformonly[\LineSpace]{my north west lines}{\pgfqpoint{-1pt}{-1pt}}{\pgfqpoint{\LineSpace}{\LineSpace}}{\pgfqpoint{\LineSpace}{\LineSpace}}%
{
  \pgfsetcolor{\tikz@pattern@color}
  \pgfsetlinewidth{0.4pt}
  \pgfpathmoveto{\pgfqpoint{0pt}{\LineSpace}}
  \pgfpathlineto{\pgfqpoint{\LineSpace + 0.1pt}{-0.1pt}}
  \pgfusepath{stroke}
}

\newdimen\LineSpace
\tikzset{
  line space/.code={\LineSpace=#1},
  line space=3pt
}
\makeatother
%%/Custom diagonal line patterns


%% Forces grid step to be the unit length by default
\tikzset{step=1}


